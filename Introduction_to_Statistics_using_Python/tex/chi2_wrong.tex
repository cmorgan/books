One example could be that we want to know whether the weight of a set of eight apples is normally distributed. Chi-square distribution can be used to test for this. Assume that the apples weigh 88, 93, 110, 76, 78, 121, 92 and 86 grams, and we have knowledge of the mean and the standard deviation weight of all apples. We obtain the normally distributed Z values by subtracting the mean weight (93) and divide by the standard deviation (15.41). For example, the first apple has got $Z_1 = \frac{88-93}{15.41} = -0.3245$ using four decimal points. Square all the  Z values, then taking the sum yields a Chi-squared distributed random variable with mean 8 and variance 16.

Now when we have the value of the chi-square statistic ''Y'', we compare it to the critical value of the chi-square distribution at n = 8 degrees of freedom and 95\% level of significance which can found in a Chi-square statistical table. The null hypothesis is that the sample of apples is normally distributed. It is rejected if the value of the test statistic is higher than the critical value.
